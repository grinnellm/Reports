In 2017, biological samples were collected by the ``Nita Maria'' in Big Bay and by the ``Franciscan No.1'' in Kitkatla.
The ``Nita Maria'' sampled for 13 days, from March 15\textsuperscript{th} to March 27\textsuperscript{th}, and the ``Franciscan No.1'' operated for a total of 13 days, from March 14\textsuperscript{th} to March 26\textsuperscript{th}.
The primary purpose of the test charter vessels was to collect biological samples from main bodies of herring from Big Bay and Kitkatla, identified from soundings.
Both vessels were also used as management platforms for the seine roe (``Franciscan No.1'', Kitkatla) and gillnet roe (``Nita Maria'', Big Bay) fisheries.
Herring spawn locations were primarily identified with fixed-wing overflights conducted by DFO Resource Management Area staff.
Four flights were conducted this season, February-April.
The dive charter vessel, ``Royal Pride'', operated a 20-day charter from March 27\textsuperscript{th} to April 15\textsuperscript{th}, surveying spawn throughout the stock area.
All three charter vessels were funded by DFO, through a contract to the Herring Conservation Research Society.