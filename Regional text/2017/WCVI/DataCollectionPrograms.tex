Biological samples were collected by the seine charter vessel ``Proud Canadian'' for 20 days from February 20\textsuperscript{th} to March 11\textsuperscript{th}.
The primary purpose of the test charter vessel was to collect biological samples from main bodies of herring from Areas 23, 24, and 25.
Nearshore herring samples were collected by the Nuu-chah-nulth staff in Area 23 as part of a pilot sampling program (a collaboration between WCVI First Nations and DFO). 
These nearshore biological samples were collected from spawning aggregations using cast nets.
WCVI First Nations were involved in the collection of spawn observations via surface surveys (in Hesquiat Harbour, Area 24), reporting spawning locations in Areas 23, 24, and 25, and the collection of biological samples in Area 24.

Herring spawn locations were primarily identified with fixed-wing overflights conducted by DFO Resource Management Area staff.
Twelve flights were conducted this season from February to April.
Two dive charter vessels operated on the WCVI.
The charter vessel ``Canadian Shore'' surveyed 15 days from March 5\textsuperscript{th} to March 20\textsuperscript{th}.
This vessel mainly covered Areas 24 and 25.
The charter vessel ``Seaveyor'', surveyed 17 days from late February through early April.
The ``Seaveyor'' covered Areas 23, 24, 25, and 27 along with a couple days in the Strait of Georgia.
All three charter vessels were funded by DFO, through a contract to the Herring Conservation Research Society.