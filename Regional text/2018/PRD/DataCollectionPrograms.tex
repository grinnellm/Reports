In 2017, biological samples were collected by two seine test vessel.
The ``Nita Maria'' collected samples in Big Bay for 13 days from March \nth{15} to March \nth{27}.
The ``Franciscan No.1'' collected samples in Kitkatla for 17 days from March \nth{15} to April \nth{3}.
The primary purpose of the test charter vessels was to collect biological samples from main bodies of herring from Big Bay and Kitkatla, identified from soundings.
Both vessels were also used as management platforms for the seine roe (``Franciscan No.1'', Kitkatla) and gillnet roe (``Nita Maria'', Big Bay) fisheries.
Biological samples were also collected by the seine vessel ``Royal Pride.''

Herring spawn locations were primarily identified with fixed-wing overflights conducted by DFO Resource Management Area staff.
Six flights were conducted this season, from February to April.
The dive charter vessel, ``Royal Pride'', operated a 20-day charter from March \nth{30} to April \nth{18}, surveying spawn throughout the stock area.
All three charter vessels were funded by DFO, through a contract to the Herring Conservation Research Society.