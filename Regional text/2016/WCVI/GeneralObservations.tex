\begin{itemize}
\item Area 23: Spawn was more spread out though the shoreline and islands of the northwest side of Barkley Sound compared with 2015.
\item Area 25: Spawn observed in Nootka Sound throughout the Spanish Pilot Group islands which are not easily observed from the regular flight path.
\item Area 25: Spawns on the outside of Nootka Island (Beano in 2015, Bajo in 2016) are difficult to find a weather window to survey due to their exposed location.
\item Area 25: The test vessel caught unspawned herring in Esperanza Inlet after the small spawn in Nuchatlitz occurred, however no further spawning was detected in Esperanza Inlet.
In addition to the spawn flights, the First Nation charter was on the grounds looking for spawn and staff from the NTC and Nuchatlaht First Nation also spent many days on the water looking for spawning activity.
\item Area 24: A spawn was observed at Whitesand Cove by the First Nation Charter with reported spawn on bough harvest but when the Seaveyor arrived
to survey the area they didn’t find any spawn.
\item Significant numbers ($\geq20$) of gray whales were observed on multiple flights feeding in Hesquiat Harbour.
On one flight, an estimated 100 gray whales were observed around Bajo Reefs after the herring spawning event in the area.
Whales were observed infrequently in Barkley Sound in 2016.
\item Area 24: Surface surveys in Hesquiat Harbour were performed on Jan 7\textsuperscript{th}, Feb 1\textsuperscript{st}, Feb 7\textsuperscript{th}, Feb 16\textsuperscript{th}, and Mar 1\textsuperscript{st}. 
These represent some of the earliest recorded spawning events, however, the Hesquiaht First Nation reports these early spawning events as `not uncommon'.
DFO records do show a number of early spawns in Hesquiat starting from mid-January into late February, with the earliest spawn in DFO records observed on January 16\textsuperscript{th}. 
\item Area 24: Deep spawn was observed on two transects on Verdia Island (Area 24) below 15\,m to 20\,m.
\end{itemize}