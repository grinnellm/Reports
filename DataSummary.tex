%##############################################################################
% 
% Author:       Matthew H. Grinnell
% Affiliation:  Pacific Biological Station, Fisheries and Oceans Canada (DFO) 
% Group:        Inshore Assessment Section, Fish Population Assessment Division
% Address:      3190 Hammond Bay Road, Nanaimo, BC, Canada, V9T 6N7
% Contact:      e-mail: matt.grinnell@dfo-mpo.gc.ca | tel: 250.756.7055
% Project:      Herring
% Code name:    DataSummary.rnw
% Version:      1.0
% Date started: Jun 03, 2016
% Date edited:  Jul 10, 2018
% 
% Overview: 
% Generate annual preliminary data summary reports for Pacific herring by region 
% using this skeleton document and:
%   The custom style file: Document/HerringSummary.sty
%   Various R objects from the .RData file generated by 'Summary.R'.
%   Child documents: Document/Variables.rnw, and Document/Context.rnw.
%   Tables and figures inported from R output (*.tex ,and *.pdf, respectively).
%   Text files in /{Year}/{Region}/*.tex.
% 
% Requirements: 
% Links to the directory that contains output from the R script 'Summary.R',
% and links to the auxiliary text files /Year/Region/*.tex.
% 
% Notes: 
% This is a dynamic document (http://yihui.name/knitr/). All of the reports can
% be made at once by sourcing the R script 'MakeDataSummaries.R'
%
%##############################################################################

% Set document style and font size
\documentclass[12pt]{article}\usepackage[]{graphicx}\usepackage[]{color}
%% maxwidth is the original width if it is less than linewidth
%% otherwise use linewidth (to make sure the graphics do not exceed the margin)
\makeatletter
\def\maxwidth{ %
  \ifdim\Gin@nat@width>\linewidth
    \linewidth
  \else
    \Gin@nat@width
  \fi
}
\makeatother

\definecolor{fgcolor}{rgb}{0.345, 0.345, 0.345}
\newcommand{\hlnum}[1]{\textcolor[rgb]{0.686,0.059,0.569}{#1}}%
\newcommand{\hlstr}[1]{\textcolor[rgb]{0.192,0.494,0.8}{#1}}%
\newcommand{\hlcom}[1]{\textcolor[rgb]{0.678,0.584,0.686}{\textit{#1}}}%
\newcommand{\hlopt}[1]{\textcolor[rgb]{0,0,0}{#1}}%
\newcommand{\hlstd}[1]{\textcolor[rgb]{0.345,0.345,0.345}{#1}}%
\newcommand{\hlkwa}[1]{\textcolor[rgb]{0.161,0.373,0.58}{\textbf{#1}}}%
\newcommand{\hlkwb}[1]{\textcolor[rgb]{0.69,0.353,0.396}{#1}}%
\newcommand{\hlkwc}[1]{\textcolor[rgb]{0.333,0.667,0.333}{#1}}%
\newcommand{\hlkwd}[1]{\textcolor[rgb]{0.737,0.353,0.396}{\textbf{#1}}}%
\let\hlipl\hlkwb

\usepackage{framed}
\makeatletter
\newenvironment{kframe}{%
 \def\at@end@of@kframe{}%
 \ifinner\ifhmode%
  \def\at@end@of@kframe{\end{minipage}}%
  \begin{minipage}{\columnwidth}%
 \fi\fi%
 \def\FrameCommand##1{\hskip\@totalleftmargin \hskip-\fboxsep
 \colorbox{shadecolor}{##1}\hskip-\fboxsep
     % There is no \\@totalrightmargin, so:
     \hskip-\linewidth \hskip-\@totalleftmargin \hskip\columnwidth}%
 \MakeFramed {\advance\hsize-\width
   \@totalleftmargin\z@ \linewidth\hsize
   \@setminipage}}%
 {\par\unskip\endMakeFramed%
 \at@end@of@kframe}
\makeatother

\definecolor{shadecolor}{rgb}{.97, .97, .97}
\definecolor{messagecolor}{rgb}{0, 0, 0}
\definecolor{warningcolor}{rgb}{1, 0, 1}
\definecolor{errorcolor}{rgb}{1, 0, 0}
\newenvironment{knitrout}{}{} % an empty environment to be redefined in TeX

\usepackage{alltt}

% Style file for the data summaries
\usepackage{Document/HerringSummary}



% Load R variables, switches, etc

% Get some values for all regions (from the saved R image)
\newcommand{\regionName}{Strait of Georgia}
\newcommand{\regionType}{major}
\newcommand{\firstYr}{1951}
\newcommand{\thisYr}{2018}
\newcommand{\fishName}{Pacific Herring}
\newcommand{\scienceName}{\emph{Clupea pallasii}}

% Get some values for CC region (from the saved R image)
\newcommand{\histYrs}{}
\newcommand{\histRat}{}
\newcommand{\fixYrs}{}

% Note if there is only a subset of sections
\newcommand{\isSubset}{}

% Initialize switches; set to TRUE/FALSE in R based on conditions in the region
\newtoggle{biosamples}  % Show biosample locations
\toggletrue{biosamples}
\newtoggle{weightCatch}  % Show weight by catch type
\toggletrue{weightCatch}
\newtoggle{weightGroup}  % Show weight by group
\togglefalse{weightGroup}
\newtoggle{spawnDepth}  % Show depth of spawn
\togglefalse{spawnDepth}
\newtoggle{numPropWtAge}  % Show number-, proportion- and weight-at-age
\togglefalse{numPropWtAge}
\newtoggle{catchStatArea}  % Show catch by stat area
\togglefalse{catchStatArea}
\newtoggle{spawnByLoc}  % Show spawn locations
\toggletrue{spawnByLoc}
\newtoggle{spawnByLocXY}  % Show spawn locations with X and Y
\toggletrue{spawnByLocXY}
\newtoggle{spatialGroup}  % Show table of stat areas, sections, and groups
\toggletrue{spatialGroup}

% Location of the pdf figures (generated in R)
\graphicspath{{../DataSummaries/SoG/}}

% Some common notes and comments
\newcommand{\sampleSize}{Each sample is approximately 100 fish.}
\newcommand{\seineSamples}{Biological summaries only include samples collected using seine nets (commercial and test) due to size-selectivity of other gear types such as gillnet.}
\newcommand{\repSamples}{Only representative biological samples are included, where `representative' indicates whether the \fishName{} sample in the set accurately reflects the larger \fishName{} school.}
\newcommand{\plusGroup}{The age-10 class is a `plus group' which includes fish ages 10 and older.}
\newcommand{\spawnIndex}{The `spawn index' represents the raw survey data only, and is not scaled by the spawn survey scaling parameter, $q$.}
\newcommand{\trendLine}{The thick black line is a loess curve, and the shaded area is the 90\% confidence interval.}
\newcommand{\qPeriods}{The spawn index has two distinct periods defined by the dominant survey method: surface surveys (1951 to 1987), and dive surveys (1988 to 2018).}
\newcommand{\legendNear}{Legend: `Nearshore' refers to samples collected using cast nets as part of a pilot study with WCVI First Nations.}
\newcommand{\legendGear}{Legend: `Other' represents the reduction, the food and bait, as well as the special use fishery; `RoeSN' represents the roe seine fishery; and `RoeGN' represents the roe gillnet fishery.}
\newcommand{\darkThisYr}{The year \thisYr{} has a darker bar to facilitate interpretation.}
\newcommand{\withPrivacy}{Note: `WP' indicates that data are withheld due to privacy concerns.}
\newcommand{\noSOK}{Data from the spawn-on-kelp (SOK) fishery is not included.}
\newcommand{\spawnIndexTechReport}{See the \href{https://github.com/grinnellm/HerringSpawnDocumentation/blob/master/SpawnIndexTechnicalReport.pdf}{draft spawn index techincal report} for calculations to convert SOK harvest to spawning biomass.}
\newcommand{\missingSpawn}[1]{Missing spawn index values (#1) indicate incomplete spawn surveys.}
\newcommand{\spawnTypes}{There are three types of spawn survey observations: observations of spawn taken from the surface usually at low tide, underwater observations of spawn on giant kelp, Macrocystis (\emph{Macrocystis} spp.), and underwater observations of spawn on other types of algae and the substrate, which we refer to as `understory.'}

% TODO: Wording in text and captions for 'tabBioNum' and 'tabBioTypeNum' re showing 'Representative' biological samples only (once the R script is updated). Use the command '\repSamples' below. This could also include the figure 'BioLocations.pdf' (Note that this creates issues with WCVI because of the cast-net biosamples, which are not representative, but are included in the data summary.)
% TODO: Active voice.
% TODO: Once the JS summary is sent/done, move text from 'Context.rnw' to this file.
% TODO: Add a note re SOK removals/mortality not included in catch (search for SOK).
% TODO: Add a table with weight- and/or length-at-age differences with respect to some reference period (i.e., 1980s).
% TODO: Remove comment about egregia SOK excluded from product weight once egregia is included in the SOK table.

% Begin the document
\IfFileExists{upquote.sty}{\usepackage{upquote}}{}
\begin{document}

% Add the title etc
\frontmatter

% Context section
\section{Context}

% Bring in stock text for context, with an update form \theFollowing

\fishName{} (\scienceName{}) in British Columbia are assessed as 5 major and 2 minor stock assessment regions (SARs), and data are collected and summarized on this scale (\autoref{tabRegions}, \autoref{figBC}).
%One difference between major and minor SARs is that only major SARs receive stock assessments.
The \fishName{} data collection program includes fishery-dependent and -independent data from \firstYr{} to \thisYr{}.
This includes annual time series of commercial catch data, biological samples (providing information on proportion-at-age and weight-at-age), and spawn index data conducted using a combination of surface and SCUBA surveys.
In some areas, industry- and/or First Nations-operated in-season soundings programs are also conducted, and this information is used by resource managers, First Nations, and stakeholders to locate fish and identify areas of high and low \fishName{} biomass to plan harvesting activities.
In-season acoustic soundings are not used by stock assessment to inform the estimation of spawning biomass.

The following is a description of data collected for \fishName{} in \thisYr{} in the \regionName{} \regionType{} SAR (\autoref{figRegion}).
Data collected outside the SAR boundary are not included in this summary, and are not used for the purposes of stock assessment.
Although we summarise data at the scale of the SAR for stock assessments, we summarise data at finer spatial scales in this report: locations are nested within sections, sections are nested within statistical areas, and statistical areas are nested within SARs (\autoref{tabSpatialGroup}).
\iftoggle{spatialGroup}{For the \regionName{} \regionType{} SAR, we use another level of spatial aggregation which we refer to as a `group'.}{\unskip}
Note that we refer to `year' instead of `herring season' in this report; therefore \thisYr{} refers to the 2017/2018 \fishName{} season.

% Data collection programs section
\section{Data collection programs}

% Bring in custom text for data collection programs
\inputsp{RegionalText/2018/SoG/DataCollectionPrograms.tex}

% Bring in stock text for additional biosamples in CC StatArea 08, if required


% Catch and biological samples section
\section{Catch and biological samples}

In the 1950s and 1960s, the reduction fishery dominated \fishName{} catch; starting in the 1970s, catch has been predominantly from roe seine and gillnet fisheries.
The reduction fishery is different from current fisheries in several ways.
First, the reduction fishery caught \fishName{} of all ages, whereas current fisheries target spawning (i.e., mature) fish.
Thus, reduction fisheries included age-1 fish which are not typically caught in current fisheries.
Second, the reduction fishery has some uncertainty regarding the quantity and location of catch; in some cases this may affect our ability to allocate catch to a specific SAR.
For the roe gillnet fishery, all \fishName{} catch has been validated by a dockside monitoring program since 1998; the catch validation program started in 1999 for the roe seine fishery.
Finally, the reduction fishery operated during the winter months, whereas roe fisheries typically target spawning fish between February and April.

% Bring in custom text for catch and biological samples
\inputsp{RegionalText/2018/SoG/CatchAndBiologicalSamples.tex}
Landed commercial catch of \fishName{} by year and fishery is shown in \autoref{tabCatchCommUseYr} and \autoref{figCatchGear}.
\iftoggle{catchStatArea}{In addition to annual catch variability, catch varies among Statistical Areas (\autoref{figCatchStatArea}).}{\unskip}
Total harvested spawn on kelp (SOK) in \thisYr{} in the \regionName{} \regionType{} SAR is shown in \autoref{tabHarvestSOK}; we also calculate the estimated spawning biomass associated with SOK harvest.
\spawnIndexTechReport{}
% Note that we do not account for SOK harvest removals in stock assessments. OR
% Currently, catch input to the stock assessment model does not include mortality from the commercial SOK fishery, nor any recreational or FSC fisheries.

In \thisYr{}, 144 \fishName{} biological samples were collected and processed for the \regionName{} \regionType{} SAR (\autoref{tabBioNum}, \autoref{tabBioTypeNum}), and a total of 8,545 \fishName{} were aged in \thisYr{}. 
\iftoggle{numPropWtAge}{Differences between biological data collected from two sampling protocols regarding the number-, proportion-, weight-, and length-at-age for \fishName{} in \thisYr{} in the \regionName{} \regionType{} SAR are shown in \autoref{tabDeltaNumAgeYr}, \autoref{tabDeltaPropAgeYr}, \autoref{tabDeltaWtAgeYr}, and \autoref{tabDeltaLenAgeYr}, respectively.
The nearshore sampling program is a multi-year pilot study (using cast nets), therefore only biological data from the seine samples were used for the purposes of stock assessment.}{\unskip}
\iftoggle{biosamples}{The locations in which the biological samples were collected are presented in \autoref{figBioLocations}.}{\unskip}
\iftoggle{weightCatch}{Biological samples collected using seine gear shows that there is considerable variability in fish weight by year and sample type (\autoref{figWeightCatch}).}{\unskip}
Included herein are biological summaries of observed proportion-, number-, weight-, and length-at-age (\autoref{figProportionAged}, \autoref{tabPropAgedYearTab}, and \autoref{figWtLenAge}, respectively).
\iftoggle{weightGroup}{Some Statistical Areas tend to have larger fish at a given age (\autoref{figWeightAgeGroup}, \autoref{tabWeightAgeGroupN}).}{\unskip}
\seineSamples{}
\repSamples{}

% Spawn survey data section
\section{Spawn survey data}

\iftoggle{spawnByLoc}{\fishName{} spawn surveys were conducted at 32 individual locations in \thisYr{} in the \regionName{} \regionType{} SAR (\autoref{tabSpawnByLoc}\iftoggle{spawnByLocXY}{, and \autoref{figSpawnByLoc}}{\unskip}).}
{No spawn surveys were conducted in \thisYr{} in the \regionName{} \regionType{} SAR.}
A summary of spawn from the last decade (2008 to 2017) is shown in \autoref{figSpawnDecade}.
\autoref{figSpawnTiming} shows spawn start date by decade and Group.
Spawn surveys are conducted to estimate the spawn length, width, number of egg layers, and substrate type, and these data are used to estimate the index of spawning biomass (i.e., the spawn index; \autoref{figSpawnIndexType}, \autoref{figSpawnDimensions}, \autoref{figSpawnIndexPercent}, \autoref{tabSpawnYrTab}, and \autoref{figSpawnIndexChange}).
We describe the calculations used to estimate the spawn index in the \href{https://github.com/grinnellm/HerringSpawnDocumentation/blob/master/SpawnIndexTechnicalReport.pdf}{draft spawn index techincal report}.
\iftoggle{spawnDepth}{In addition, spawn surveys estimate spawn depth by Statistical Area, and Section (\autoref{figSpawnDepth}).}{\unskip}
\spawnIndex{}
Therefore, these data do not represent model estimates of spawning biomass, and are considered the minimum observed spawning biomass derived from egg counts.
\qPeriods{}

Some \fishName{} Sections contribute more than others to the total spawn index, and the percentage contributed by Section varies yearly (\autoref{figSpawnIndexPercent}b, \autoref{figSpawnPercentPanel}). 
For example, in \thisYr{}, Section 143 contributed the most to the spawn index (88\%). 
As with Sections, some Groups contribute more than others to the total spawn index (\autoref{figSpawnIndexPercent}c, \autoref{figSpawnPercentGrid}).
An animation shows the spawn index by spawn survey location from 1951 to \thisYr{} (\autoref{figSpawnIndexAnimation}).

% General observations section
\section{General observations}

General observations provide context to the data summary report.
The following observations were reported by area DFO Resource Management staff, and DFO Science staff:

% Bring in custom text, usually a bullet list
\inputsp{RegionalText/2018/SoG/GeneralObservations.tex}

% Start a new page, and then place the tables
\clearpage
\section{Tables}

\begin{table}[h]
\centering
\caption{\fishName{} stock assessment regions (SARs) in British Columbia.} 
\begin{tabular}{lll}
\toprule
\inputsp{../DataSummaries/SoG/Regions.tex}
\end{tabular}
\label{tabRegions}
\end{table}

\FloatBarrier
\begin{longtable}{\iftoggle{spatialGroup}{lrrl}{lrr}}
\caption{Statistical Areas\iftoggle{spatialGroup}{, Sections, and Groups}{ and Sections} for \fishName{} in the \regionName{} \regionType{} stock assessment region (SAR).
Legend: `14\&17' is Statistical Areas 14 and 17
      (excluding Section 173); `ESoG' is eastern Strait of Georgia; 
      `Lazo' is above Cape Lazo; and `SDodd' is South of Dodd Narrows}\\
\toprule
Region & Statistical Area & Section & Group\\
\midrule
\endfirsthead
\toprule
\multicolumn{\iftoggle{spatialGroup}{4}{3}}{c}{\emph{\tablename\ \thetable{} continued}}\\
Region & Statistical Area & Section & Group\\
\midrule
\endhead
\bottomrule
\endfoot
\bottomrule
\endlastfoot
\inputsp{../DataSummaries/SoG/SpatialGroup.tex}%
\label{tabSpatialGroup}
\end{longtable}
\FloatBarrier

\begin{table}[h]
\centering
\caption{Total landed commercial catch of \fishName{} in metric tonnes (t) by gear type in \thisYr{} in the \regionName{} \regionType{} stock assessment region (SAR).
\legendGear{}
\noSOK{}
\withPrivacy{}}
\begin{tabular}{lr}
\toprule
\inputsp{../DataSummaries/SoG/CatchCommUseYr.tex}
\end{tabular}
\label{tabCatchCommUseYr}
\end{table}

\begin{table}[h]
\centering
\caption{Total harvested \fishName{} spawn on kelp (SOK) in pounds (lb), and the associated estimate of spawning biomass in metric tonnes (t) from 2008 to \thisYr{} in the \regionName{} \regionType{} stock assessment region (SAR).
\spawnIndexTechReport{}
%
\withPrivacy{}}
\begin{tabular}{rrr}
\toprule
\inputsp{../DataSummaries/SoG/HarvestSOK.tex}
\end{tabular}
\label{tabHarvestSOK}
\end{table}

\begin{table}[h]
\centering
\caption{Number of \fishName{} biological samples processed from 2008 to \thisYr{} in the \regionName{} \regionType{} stock assessment region (SAR).
\sampleSize{}} 
\begin{tabular}{rrrr}
\toprule
& \multicolumn{3}{c}{Number of samples}\\
\cmidrule{2-4}
\inputsp{../DataSummaries/SoG/BioNum.tex}
\end{tabular}
\label{tabBioNum}
\end{table}

\begin{table}[h]
\centering
\caption{Number and type of \fishName{} biological samples processed in \thisYr{} in the \regionName{} \regionType{} stock assessment region (SAR). 
\sampleSize{}} 
\begin{tabular}{lllr}
\toprule
\inputsp{../DataSummaries/SoG/BioTypeNum.tex}
\end{tabular}
\label{tabBioTypeNum}
\end{table}

\iftoggle{numPropWtAge}{
\begin{table}[h]
\centering
\caption{Observed number-at-age of \fishName{} by sample type in \thisYr{} in the \regionName{} \regionType{} stock assessment region (SAR). 
\plusGroup{}
\legendNear{}} 
\begin{tabular}{lrrrrrrrrr}
\toprule
& \multicolumn{9}{c}{Number-at-age}\\
\cmidrule{2-10}
\inputsp{../DataSummaries/SoG/DeltaNumAgeYr.tex}
\end{tabular}
\label{tabDeltaNumAgeYr}
\end{table}

\begin{table}[h]
\centering
\caption{Observed proportion-at-age of \fishName{} by sample type in \thisYr{} in the \regionName{} \regionType{} stock assessment region (SAR). 
\plusGroup{}
\legendNear{}} 
\begin{tabular}{lrrrrrrrrr}
\toprule
& \multicolumn{9}{c}{Proportion-at-age}\\
\cmidrule{2-10}
\inputsp{../DataSummaries/SoG/DeltaPropAgeYr.tex}
\end{tabular}
\label{tabDeltaPropAgeYr}
\end{table}

\begin{table}[h]
\centering
\caption{Observed mean weight-at-age in grams (g) of \fishName{} by sample type in \thisYr{} in the \regionName{} \regionType{} stock assessment region (SAR). 
\plusGroup{}
\legendNear{}} 
\begin{tabular}{lrrrrrrrrr}
\toprule
& \multicolumn{9}{c}{Mean weight-at-age (g)}\\
\cmidrule{2-10}
\inputsp{../DataSummaries/SoG/DeltaWtAgeYr.tex}
\end{tabular}
\label{tabDeltaWtAgeYr}
\end{table}

\begin{table}[h]
\centering
\caption{Observed mean length-at-age in millimetres (mm) of \fishName{} by sample type in \thisYr{} in the \regionName{} \regionType{} stock assessment region (SAR). 
\plusGroup{}
\legendNear{}} 
\begin{tabular}{lrrrrrrrrr}
\toprule
& \multicolumn{9}{c}{Mean length-at-age (mm)}\\
\cmidrule{2-10}
\inputsp{../DataSummaries/SoG/DeltaLenAgeYr.tex}
\end{tabular}
\label{tabDeltaLenAgeYr}
\end{table}
}{\unskip}

\begin{table}[h]
\centering
\caption{Observed proportion-at-age for \fishName{} from 2008 to \thisYr{} in the \regionName{} \regionType{} stock assessment region (SAR). 
\plusGroup{}} 
\begin{tabular}{lrrrrrrrrr}
\toprule
& \multicolumn{9}{c}{Proportion-at-age}\\
\cmidrule{2-10}
\inputsp{../DataSummaries/SoG/PropAgedYearTab.tex}
\end{tabular}
\label{tabPropAgedYearTab}
\end{table}

\iftoggle{weightGroup}{
\begin{table}[h]
\centering
\caption{Sample size for \fishName{} weight-at-age analysis in \thisYr{} in the \regionName{} \regionType{} stock assessment region (SAR) by Group from the most recent decade (), and the previous decade (), as displayed in \autoref{figWeightAgeGroup}. 
\seineSamples{}
\plusGroup{}
Legend: `14\&17' is Statistical Areas 14 and 17
      (excluding Section 173); `ESoG' is eastern Strait of Georgia; 
      `Lazo' is above Cape Lazo; and `SDodd' is South of Dodd Narrows}
\begin{tabular}{lrrr}
\toprule
& & \multicolumn{2}{c}{Sample size}\\
\cmidrule{3-4}
\inputsp{../DataSummaries/SoG/WeightAgeGroupN.tex}
\end{tabular}
\label{tabWeightAgeGroupN}
\end{table}
}{\unskip}

\iftoggle{spawnByLoc}{
\FloatBarrier
\begin{longtable}{rrllr}
\caption{\fishName{} spawn survey locations, start date, and spawn index in metric tonnes (t) in \thisYr{} in the \regionName{} \regionType{} stock assessment region (SAR).
\spawnIndex{}
\missingSpawn{NAs}}\\
\toprule
Statistical Area & Section & Location name & Start date & Spawn index (t)\\ 
\midrule
\endfirsthead
\toprule
\multicolumn{5}{c}{\emph{\tablename\ \thetable{} continued}}\\
Statistical Area & Section & Location name & Start date & Spawn index (t)\\ 
\midrule
\endhead
\bottomrule
\endfoot
\bottomrule
\endlastfoot
\inputsp{../DataSummaries/SoG/SpawnByLoc.tex}%
\label{tabSpawnByLoc}
\end{longtable}
\FloatBarrier
}{\unskip}

\begin{table}[h]
\centerfloat
\caption{Summary of \fishName{} spawn survey data from 2008 to \thisYr{} in the \regionName{} \regionType{} stock assessment region (SAR). 
\qPeriods{}
\spawnIndex{}
Units: metres (m), and metric tonnes (t).}
\begin{tabular}{rrrrr}
\toprule
\inputsp{../DataSummaries/SoG/SpawnYrTab.tex}
\end{tabular}
\label{tabSpawnYrTab}
\end{table}

% Start a new page, and then place the figures
\clearpage
\section{Figures}

\begin{figure}[h]
\centering
\includegraphics[width=\linewidth]{BC.pdf}
\caption{Boundaries for the \fishName{} stock assessment regions (SARs) in British Columbia: there are 5 major SARs (HG, PRD, CC, SoG, and WCVI), and 2 minor SARs (A27 and A2W; \autoref{tabRegions}).
Units: kilometres (km).}
\label{figBC}
\end{figure}

\begin{figure}[h]
\centering
\includegraphics[width=\linewidth]{Region.pdf}
\caption{Boundaries for the \regionName{} \regionType{} stock assessment region (SAR; thick dashed lines), associated Statistical Areas (SA; thin solid lines), and associated Sections (thin dotted lines).
Units: kilometres (km).
Legend: `14\&17' is Statistical Areas 14 and 17
      (excluding Section 173); `ESoG' is eastern Strait of Georgia; 
      `Lazo' is above Cape Lazo; and `SDodd' is South of Dodd Narrows}
\label{figRegion}
\end{figure}

\begin{figure}[h]
\centering
\includegraphics[width=\linewidth]{CatchGear.pdf}
\caption{Time series of total landed catch in thousands of metric tonnes ($\text{t} \times 10^{3}$) of \fishName{} by gear type from 1951 to \thisYr{} in the \regionName{} \regionType{} stock assessment region (SAR).
\legendGear{}
\noSOK{}}
\label{figCatchGear}
\end{figure}

\iftoggle{catchStatArea}{
\begin{figure}[h]
\centering
\includegraphics[width=\linewidth]{CatchStatArea.pdf}
\caption{Time series of total landed catch in thousands of metric tonnes ($\text{t} \times 10^{3}$) of \fishName{} by Statistical Area (SA) from 1983 to \thisYr{} in the \regionName{} \regionType{} stock assessment region (SAR).
\darkThisYr{}}
\label{figCatchStatArea}
\end{figure}
}{\unskip}

\iftoggle{biosamples}{
\begin{figure}[h]
\centering
\includegraphics[width=\linewidth]{BioLocations.pdf}
\caption{Location and type of \fishName{} biological samples collected in \thisYr{} in the \regionName{} \regionType{} stock assessment region (SAR; thick dashed lines), and associated Sections (Sec; thin solid lines).
Units: kilometres (km).}
\label{figBioLocations}
\end{figure}
}{\unskip}

\iftoggle{weightCatch}{
\begin{figure}[h]
\centering
\includegraphics[width=\linewidth]{WeightCatch.pdf}
\caption{Time series of weight in grams (g) of \fishName{} by sample type from 2008 to \thisYr{} in the \regionName{} \regionType{} stock assessment region (SAR) in Statistical Areas 14 and 17.
The outer edges of the boxes indicate the \nth{25} and \nth{75} percentiles, and the middle lines indicate the \nth{50} percentiles (i.e., medians).
The whiskers extend to $1.5 \times \text{IQR}$, where IQR is the distance between the \nth{25} and \nth{75} percentiles, and dots indicate outliers.
Horizontal dashed lines indicate the mean weight-at-age for age-2 (lowest line) to age-10 (incrementing higher from age-2) fish. 
\seineSamples{}
\plusGroup{}}
\label{figWeightCatch}
\end{figure}
}{\unskip}

\begin{figure}[h]
\centering
\includegraphics[width=\linewidth]{ProportionAged.pdf}
\caption{Time series of observed proportion-at-age (a) and number aged in thousands (c) of \fishName{} from 1951 to \thisYr{} in the \regionName{} \regionType{} stock assessment region (SAR). 
The black line is the mean age, and the shaded area is the approximate 90\% distribution. 
\seineSamples{}
\plusGroup{}}
\label{figProportionAged}
\end{figure}

\begin{figure}[h]
\centering
\includegraphics[width=\linewidth]{WtLenAge.pdf}
\caption{Time series of weight-at-age in grams (g; panel a) and length-at-age in milimetres (mm; panel b) for age-3 (circles) and 5-year running mean weight- and length-at-age (lines) for \fishName{} from 1951 to \thisYr{} in the \regionName{} \regionType{} stock assessment region (SAR). 
% Lines show nRoll-year running means for age-min(ageRange) to age-max(ageRange) fish (incrementing higher from the lowest line); the thick black line highlights age-ageShow fish.
Missing weight- and length-at-age values (i.e., years with no biological samples) are imputed using one of two methods: missing values at the beginning of the time series are imputed by extending the first non-missing value backwards; other missing values are imputed as the mean of the previous 5 years.
\seineSamples{}
\plusGroup{}}
\label{figWtLenAge}
\end{figure}

\iftoggle{weightGroup}{
\begin{figure}[h]
\centering
\includegraphics[width=\linewidth]{WeightAgeGroup.pdf}
\caption{Weight-at-age in grams (g) of \fishName{} in the \regionName{} \regionType{} stock assessment region (SAR) by Group from the most recent decade (), and the previous decade (). 
The outer edges of the boxes indicate the \nth{25} and \nth{75} percentiles, and the middle lines indicate the \nth{50} percentiles (i.e., medians).
The whiskers extend to $1.5 \times \text{IQR}$, where IQR is the distance between the \nth{25} and \nth{75} percentiles, and dots indicate outliers.
Sample sizes are given in \autoref{tabWeightAgeGroupN}.
\seineSamples{}
\plusGroup{}
Legend: `14\&17' is Statistical Areas 14 and 17
      (excluding Section 173); `ESoG' is eastern Strait of Georgia; 
      `Lazo' is above Cape Lazo; and `SDodd' is South of Dodd Narrows}
\label{figWeightAgeGroup}
\end{figure}
}{\unskip}

\iftoggle{spawnByLoc}{
\iftoggle{spawnByLocXY}{
\begin{figure}[h]
\centering
\includegraphics[width=\linewidth]{SpawnByLoc.pdf}
\caption{\fishName{} spawn survey locations, and spawn index in metric tonnes (t) in \thisYr{} in the \regionName{} \regionType{} stock assessment region (SAR; thick dashed lines), and associated Sections (Sec; thin solid lines).
\spawnIndex{}
\missingSpawn{grey circles}
Units: kilometres (km).}
\label{figSpawnByLoc}
\end{figure}
}{}
}{\unskip}

\begin{figure}[h]
\centering
\includegraphics[width=\linewidth]{SpawnDecade.pdf}
\caption{\fishName{} spawn survey locations, mean spawn index in metric tonnes (t), and spawn frequency from 2008 to 2017 in the \regionName{} \regionType{} stock assessment region (SAR; thick dashed lines), and associated Sections (Sec; thin solid lines).
\spawnIndex{}
\missingSpawn{grey circles}
%Note that the mean excludes years when no spawn was reported.
Units: kilometres (km).}
\label{figSpawnDecade}
\end{figure}

\begin{figure}[h]
\centering
\includegraphics[width=\linewidth]{SpawnTiming.pdf}
\caption{\fishName{} spawn start date by decade and Group.
Grey shaded regions indicate March \nth{1} to \nth{31}.
Note that spawn size and intensity varies; therefore the number of spawns is not directly proportional to spawn extent or biomass.
Legend: `14\&17' is Statistical Areas 14 and 17
      (excluding Section 173); `ESoG' is eastern Strait of Georgia; 
      `Lazo' is above Cape Lazo; and `SDodd' is South of Dodd Narrows}
\label{figSpawnTiming}
\end{figure}

\begin{figure}[h]
\centering
\includegraphics[width=\linewidth]{SpawnIndexType.pdf}
\caption{Time series of spawn index in thousands of metric tonnes ($\text{t} \times 10^{3}$) by type for \fishName{} from 1983 to \thisYr{} in the \regionName{} \regionType{} stock assessment region (SAR). 
%\trendLine{}
\spawnTypes{}
\qPeriods{}}
\label{figSpawnIndexType}
\end{figure}

\begin{figure}[h]
\centering
\includegraphics[width=\linewidth]{SpawnDimensions.pdf}
\caption{Time series of total spawn length in thousands of metres ($\text{m} \times 10^{3}$; panel a), mean spawn width in metres (b), and mean number of egg layers (c) for \fishName{} from 1983 to \thisYr{} in the \regionName{} \regionType{} stock assessment region (SAR). 
%\trendLine{}
\qPeriods{}}
\label{figSpawnDimensions}
\end{figure}

\begin{figure}[h]
\centering
\includegraphics[width=\linewidth]{SpawnIndexPercent.pdf}
\caption{Time series of spawn index in thousands of metric tonnes ($\text{t} \times 10^{3}$) for \fishName{} from 1983 to \thisYr{} in the \regionName{} \regionType{} stock assessment region (SAR; panel a), as well as percent contributed by Group, and Section (b, \& c, respectively). 
%\trendLine{}
\qPeriods{}
\spawnIndex{}
Legend: `14\&17' is Statistical Areas 14 and 17
      (excluding Section 173); `ESoG' is eastern Strait of Georgia; 
      `Lazo' is above Cape Lazo; and `SDodd' is South of Dodd Narrows}
\label{figSpawnIndexPercent}
\end{figure}

\begin{figure}[h]
\centering
\includegraphics[width=\linewidth]{SpawnIndexChange.pdf}
\caption{Time series of spawn index in thousands of metric tonnes ($\text{t} \times 10^{3}$) for \fishName{} from 1983 to \thisYr{} in the \regionName{} \regionType{} stock assessment region (SAR; panel a), and percent change (b).
We calculate percent change as $\delta_t=\frac{I_t - I_{t-1}}{I_{t-1}}$ where $\mli{I}_t$ is the spawn index in year $t$.
%\trendLine{}
\qPeriods{}
\spawnIndex{}}
\label{figSpawnIndexChange}
\end{figure}

\iftoggle{spawnDepth}{
\begin{figure}[h]
\centering
\includegraphics[width=\linewidth]{SpawnDepthSASec.pdf}
\caption{Time series of maximum spawn depth in metres (m) for \fishName{} from 1983 to \thisYr{} in the \regionName{} \regionType{} stock assessment region (SAR) by Statistical Area (SA; panel a), and Section (b).
Note that depth is not corrected to the chart datum.
\qPeriods{}}
\label{figSpawnDepth}
\end{figure}
}{\unskip}

\begin{figure}[h]
\centering
\includegraphics[width=\linewidth]{SpawnPercentPanel.pdf}
\caption{Time series of percent of spawn index by Section for \fishName{} from 1983 to \thisYr{} in the \regionName{} \regionType{} stock assessment region (SAR). 
\darkThisYr{}
\qPeriods{}
\spawnIndex{}}
\label{figSpawnPercentPanel}
\end{figure}

\begin{figure}[h]
\centering
\includegraphics[width=\linewidth]{SpawnPercentGrid.pdf}
\caption{Time series of percent of spawn index by Group and Section for \fishName{} from 1983 to \thisYr{} in the \regionName{} \regionType{} stock assessment region (SAR). 
\qPeriods{}
\spawnIndex{}
Legend: `14\&17' is Statistical Areas 14 and 17
      (excluding Section 173); `ESoG' is eastern Strait of Georgia; 
      `Lazo' is above Cape Lazo; and `SDodd' is South of Dodd Narrows}
\label{figSpawnPercentGrid}
\end{figure}

\begin{figure}[h]
\centering
\animategraphics[width=\linewidth]{1}{../DataSummaries/SoG/SpawnIndexAnimation}{}{}
\caption{Animation of \fishName{} spawn survey locations and spawn index in metric tonnes (t) from 1951 to \thisYr{} in the \regionName{} \regionType{} stock assessment region (SAR; thick dashed lines), and associated Sections (Sec; thin solid lines).
\qPeriods{}
\spawnIndex{}
\missingSpawn{grey circles}
The inset shows the total spawn index by year.
Units: kilometres (km).}
\label{figSpawnIndexAnimation}
\end{figure}

% Fin
\end{document}
