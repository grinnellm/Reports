The data collection program in the Central Coast reflected a number of collaborations between DFO, the Heiltsuk Nation, and the Herring Industry.
Biological samples were collected by two vessels, the seine test charter ``Franciscan No.1'' for 10 days from March 1\textsuperscript{st} to March 10\textsuperscript{th}, and the seine charter vessel ``Proud Canadian'' for 21 days from March 15\textsuperscript{th} to April 4\textsuperscript{rd}.
The primary purpose of the test charter vessel was to collect biological samples from main bodies of herring in Statistical Areas 06, 07, and 08, identified from soundings.
 
Herring spawn locations were primarily identified with fixed-wing overflights conducted by DFO Resource Management Area staff.
Eight flights were conducted this season, February--April.
Two dive charter vessels operated in the CC:
\begin{itemize}
\item The charter vessel ``Pachena No.1'' surveyed 21 days from April8\textsuperscript{nd} to April 28\textsuperscript{th}, 
\item The charter vessel ``Ocean Cloud'' surveyed 12 days from April 6\textsuperscript{th} to April 17\textsuperscript{th}, and 
\item The Kitasoo First Nations conducted 1 day of dive survey on spawn in Culpepper Lagoon (Area 06).
\end{itemize}
Three gillnet sounding vessels were operating by the Heiltsuk Nation this season: two primarily in Area 07 and one primarily in Area 08 to assist the location of fish for the spawn on kelp (SOK) operations.