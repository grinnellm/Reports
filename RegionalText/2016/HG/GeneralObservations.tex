\begin{itemize}
\item Observations that spawn has moved off HG proper and more into the outer islands.
\item Observations of spawning season concentrated in time (occurring over a shorter number of days), as opposed to in the past when spawning season was more spread out.
\item Warmer water termperatures were observed again in 2016, and resulted in early spawns, with the majority of spawns occurring within a 10-week period.
\item We received fewer reports of fungal eggs (when compared with 2015)---to be confirmed.
\item Area with significant egg mortality and fungal growth was observed by dive teams in areas of very thick spawn along the Burnaby Island shore in Pool Inlet.
\item Dive charter needed to skip transects in some areas so ensure sampling in all areas before eggs hatched out.
The dive team then returned to areas and surveyed missed transects once the area was completed.
After spawn was all measured, no new reports of spawn were reported.
\item Observed decline in herring spawn in 2016.
This is also reflected in the data: decline in total length, average width, with total length declining by almost half.
\item Spawns in Louscoone Inlet and Carpenter Bay were reported but were not surveyed.
General sense is the majority ($>90\%$) of  the HG spawn was surveyed.
\item FSC harvest hast not been reported however anecdotal reports indicated that kelp quality was poor this year and FSC harvests were negligible.
\item Spawns from Skidegate Inlet are monitored by HFP and have not been reported to HG Resource Manager.
\end{itemize}