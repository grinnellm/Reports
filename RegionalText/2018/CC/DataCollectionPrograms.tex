The data collection program in the Central Coast reflected a number of collaborations between DFO, the Heiltsuk Nation, and the Herring Industry.
Biological samples were collected by the seine test charter vessel ``Windward Isle'' for 21 days from March \nth{15} to April \nth{4}.
The primary purpose of the test charter vessel was to collect biological samples from main bodies of herring in Statistical Areas 06, 07, and 08, identified from soundings.
 
Herring spawn locations were primarily identified with fixed-wing overflights conducted by DFO Resource Management Area staff.
Seven flights were conducted this season, between February and April.
Three dive charter vessels operated in the CC:
\begin{itemize}
\item The ``Viking Spirit'' surveyed 21 days from April \nth{6} to April \nth{25},
\item The ``Ocean Cloud'' surveyed 12 days from April \nth{4} to April \nth{15}, and
\item The Kitasoo First Nations surveyed 5 days from April \nth{1} to April \nth{5}.
\end{itemize}
Three gillnet sounding vessels were operating by the Heiltsuk Nation this season: two primarily in Area 07 and one primarily in Area 08 to assist the location of fish for the spawn on kelp (SOK) operations.