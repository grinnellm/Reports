Biological samples were collected by the seine charter vessel ``Proud Canadian'' for 21 days from February \nth{20} to March \nth{12}.
The primary purpose of the test charter vessel was to collect biological samples from main bodies of herring from Areas 23, 24, and 25.
Nearshore herring samples were collected by the Nuu-chah-nulth staff in Areas 23, 24, and 25 as part of a pilot sampling program and on-going collaboration between WCVI First Nations and DFO.
These nearshore biological samples were collected from spawning aggregations using cast nets.
WCVI First Nations spawn reconnaissance charters reported spawn activity in all 3 Areas, and conducted spawn surveys using surface survey methods in Hesquiat Harbour (Area 24).

Herring spawn locations were primarily identified with fixed-wing overflights conducted by DFO Resource Management Area staff.
Ten flights were conducted this season from February to April.
Three dive charter vessels operated on the WCVI:
\begin{itemize}
\item The ``Double Decker'' surveyed 18 days from March \nth{6} to March \nth{23}, mainly in Areas 24 and 25,
\item The ``Seaveyor'' surveyed 11 days from March \nth{2} to March \nth{27}, and
\item The ``Viking Spirit'' surveyed 4 days from March \nth{20} to Match \nth{24}.
\end{itemize}
The seine test charter vessel and the dive survey vessels were funded by DFO, through a contract to the Herring Conservation Research Society.
