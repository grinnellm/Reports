\begin{itemize}
% From Corey
\item Similar to last year, sea surface temperatures have been cooler than the previous couple of years.
\item Herring appeared to be holding in the typical pattern along the $110\,\text{m}$ edge, as opposed to holding in the shallows as seen with warmer temperatures.
\item Herring did not spawn in a wave like they typically do; instead spawning occurred in small spot spawns.
\item The early spot spawns are not unusual; however the main spawn was later which may be associated with colder water temperatures.
\item The egg hatch out time was back to normal (2 to 3 weeks), as opposed to 10 days seen with warmer sea surface temperatures.
\item The sea lion population appeared to be increased from recent years, and humpback whales were not spotted in Kitkatla Inlet.
\item Early fish that started to spawn on the \nth{19} in Serpentine Inlet may have been in the shallow waters of the Wilcox Group, and not observed on soundings.
\item The duration of the charter was extended to include the large tides that occurred over the Easter weekend, however the later spawning fish were not observed.
\item Stocks that were sounded were extremely low, however the herring spawn length appeared to be average.
Mature herring were not observed by the sounding vessel in the typical areas, and therefore were not assessed prior to spawning.
\item One seine set was almost all juveniles ($<160\,\text{mm}$) close to Snass Point.
\end{itemize}