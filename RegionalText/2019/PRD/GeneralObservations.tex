\begin{itemize}
\item Sea surface temperatures were $\sim9^\circ\text{C}$ at Kitkatla, and $\sim6^\circ\text{C}$ degrees in Big Bay.
With the cooling temperatures, herring appear to be settling back to traditional holding patterns and spawn timing.
\item Soundings and observed spawns indicated more fish than usual in Kitkatla Inlet. 
\item Increase in overall length of spawn compared to previous years.
\item Spawning occurred around the same time in Big Bay and Kitkatla.
\item Compared to last year, spawning was generally more continuous rather than spot spawns.
However, spawn intensity was still weak compared to past years. 
\item There were large extensive spawns around Dries Inlet in Kitkatla for over a week, with spawns out around Cape George and Oona River.
Spawns were observed at the North end of Porcher Island, in Hudson's Bay Passage south of Dundas Island, and in Venn Passage outside of Metlakatla.
A significant spawn occurred outside Kitkatla Inlet and Freeman Pass, from Cape George to Joachim Point. 
Moderate spawns reported at Banks Island and Willis Bay prior to the test fishery.
\item No spawn observed in Butler's Cove.
\item Generally, the egg hatch-out time was back to normal in Kitkatla (2 to 3 weeks), as opposed to 10 days seen with warmer sea surface temperatures.
Hatch-out time around Big Bay and outside Tree Bluff was shorter (10 days).
\item Test sets in Kitkatla indicated that two waves of fish were encountered.
The first wave was present at the beginning of the test charter.
The second wave appeared to move in at the end of the test charter, with a mix of juveniles and spawned out fish.
\item Herring were observed to be generally smaller than previous years, and gradually increased in size in the Duncan bay/Tugwell Island area.
\item The sea lion population appeared to be increased from recent years.
\end{itemize}