The data collection program in the Central Coast reflected a number of collaborations between DFO, the Heiltsuk Nation, and the Herring Industry.
Biological samples were collected by the seine test charter vessel ``Windward Isle'' for 22 days from March \nth{14} to April \nth{4}.
The primary purpose of the test charter vessel was to collect biological samples from main bodies of herring in Statistical Areas 06, 07, and 08, identified from soundings.
Due to COVID-19, no nearshore cast net herring samples were collected by Heiltsuk or Kitasoo in 2020.

Herring spawn locations were primarily identified with fixed-wing overflights conducted by DFO Resource Management Area staff.
Four spawn flights were conducted this season, in late March and early April.
Two dive charter vessels operated in the CC:
\begin{itemize}
\item The ``Pachena No.1'' surveyed 17 days from April \nth{2} to April \nth{18}, and
\item The ``Ocean Cloud'' surveyed 17 days from March \nth{27} to April \nth{12}.
\end{itemize}
Three gillnet sounding vessels were operated by the Heiltsuk Nation this season:
two primarily in Statistical Area 07, and
one primarily in Statistical Area 08 to assist in locating fish for spawn on kelp (SOK) operations.