Biological samples were collected by the seine charter vessel ``Denman Isle'' for 27 days from February \nth{20} to March \nth{19}.
Three additional Industry test vessels collected biological samples between February \nth{27} to March \nth{20}.
The primary purpose of the test charter vessel was to collect biological samples from main bodies of herring in Statistical Areas 14 and 17, as identified from soundings.

Herring spawn locations were primarily identified with fixed-wing overflights conducted by DFO Resource Management Area staff.
Fifteen flights were conducted this season, between February and April.
Three dive charter vessels operated in the SoG:
\begin{itemize}
\item The ``Viking Spirit'' surveyed 21 days from March \nth{13} to April \nth{2}, 
\item The ``Ocean Cloud'' surveyed 12 days from March \nth{12} to March \nth{23}, and
\item The ``Seaveyor'' surveyed 1 day on the Sunshine Coast on March \nth{2}.
\end{itemize}
In addition, Atlegay divers surveyed Statistical Area 13 on March \nth{2}.
The dive vessels and the seine charter vessel ``Denman Isle'' were funded by DFO, through a contract to the Herring Conservation Research Society.
Additional sampling and sounding efforts conducted through the Industry Test Program were funded by the Herring Industry.