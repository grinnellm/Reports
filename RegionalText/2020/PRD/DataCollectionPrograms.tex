In 2020, biological samples were collected by two seine test vessels:
\begin{itemize}
\item The ``Nita Maria'' collected samples in Big Bay for 18 days from March \nth{15} to April \nst{1}, and
\item The ``Viking Leader'' collected samples in Kitkatla for 18 days from March \nth{16} to April \nnd{2}.
\end{itemize}
The primary purpose of the test charter vessels was to collect biological samples from main bodies of herring from Big Bay and Kitkatla, identified from soundings.
Herring spawn locations were primarily identified using drones, operated from both the ``Nita Maria'' and the ``Viking Leader''.
Due to COVID-19, the dive charter vessel ``Royal Pride'' was repurposed as a surface survey charter,
successfully conducting surface surveys for a 17 day charter from March \nth{27} to April \nve{12},
surveying spawn throughout the stock area.
All three charter vessels were funded by DFO, through a contract to the Herring Conservation Research Society.