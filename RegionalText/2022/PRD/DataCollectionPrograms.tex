In 2022, biological samples were collected by two seine test vessels:

\begin{itemize}

\item The \boat{Nita Maria} collected samples in Big Bay for 13 days
from March \nth{15} to March \nth{27}, and

\item The \boat{Viking Leader} collected samples in Kitkatla for 13 days
from March \nth{15} to March \nth{27}.

\end{itemize}

The primary purpose of the test charter vessels was to collect biological samples
from main bodies of herring from Big Bay and Kitkatla,
identified from soundings.
The dive charter vessel \boat{Royal Pride} surveyed spawn for 20 days
from March \nth{30} to April \nth{18}.
Herring spawn locations were primarily identified with fixed-wing overflights.
Two spawn flights were conducted this season on March \nth{30} and April \nth{5}.
All three charter vessels were funded by DFO,
through a contract to the Herring Conservation Research Society.
