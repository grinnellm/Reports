\begin{itemize}

\item{Area 4}

\begin{itemize}

\item Sea surface temperatures were cold at 4$^\circ\text{C}$.

\item With the colder sea surface temperatures, herring were holding and
behaving in a manner which has not been observed in recent years.

\item Kelp was more abundant from Tree Bluff south to Duncan Bay. 

\item Overall spawn was more intense over two days and was protracted.
Spawning occurred slowly over the season
opposed to the warmer sea temperatures with a short spawn and quick hatch out.

\item Herring were large and healthy.

\item Peak sounding was above average at 15k.

\item Overall length of spawn was shorter, but the layers were thicker.

\end{itemize}

\item{Area 5}

\begin{itemize}

\item Water temperature was colder than recent years but warmer than Area 4
(about 6$^\circ\text{C}$).

\item Significant and consistent storm activity
over the duration of the test fishery and dive survey.

\item Large schools observed in the inlet and tended to congregate
in large schools close to the bottom and the beaches.
Fish appeared longer and larger in the test sets than in recent years.

\item There was an observed increase in sea lion abundance and activity
throughout the area.

\item Spawn appeared to commence at the end of March and was distributed
throughout the inlet and along the outside from Cape George towards Oval Point.

\item Reports of reduced kelp coverage throughout the area.

\end{itemize}

\end{itemize}