In 2022, biological samples were collected by the seine charter vessel
\boat{Denman Isle} for 27 days from February \nth{20} to March \nth{18}.
Two additional Industry test vessels collected biological samples
between February \nth{28} to March \nth{15}.
The primary purpose of the test charter vessel was to
collect biological samples from main aggregations of herring in the SoG,
as identified from soundings within the SAR boundary.

Herring spawn locations were primarily identified with fixed-wing overflights
conducted by DFO Resource Management and Science staff.
Twelve flights were conducted this season, between February and March.

Four dive vessels operated in the SoG:

\begin{itemize}

\item The \boat{Viking Spirit} surveyed 21 days from March \nth{11} to March \nth{31},

\item The \boat{Ocean Cloud} surveyed 12 days from March \nth{9} to March \nth{20},

\item Divers with A-Tlegay Fisheries Society surveyed 1 day on March \nth{7}, and

\item DFO divers surveyed 1 day on March \nth{29}.

\end{itemize}

The first two dive vessels and the seine charter vessel \boat{Denman Isle} were funded by DFO,
through a contract to the Herring Conservation Research Society.
DFO mentored A-Tlegay Fisheries Society divers for the spawn survey protocol in 2017 and 2019;
in 2022, these divers surveyed spawn through a contract with DFO.
Additional sampling and sounding efforts conducted through the
Industry Test Program were contributed in-kind by the Herring Industry.
