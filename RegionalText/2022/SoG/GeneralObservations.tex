\begin{itemize}

\item Late February, as well as early and mid-March had strong winds
which impacted the effectiveness of the assessment program.

\item Spawn in the Strait of Georgia was observed along
the East coast of Vancouver Island from Kitty Coleman south to Union Point,
on the East coast of Denman Island from White Spit to Komas Bluff,
around most of Hornby Island,
and from Bowser to the Little Qualicum River.
South of Little Qualicum River there were small spawns
at Lock Bay on Gabriola and at Round Island south of Dodd Narrows. 

\item Additional spawn was reported in Areas 13, 15, 16, 19, and 28.

\item Spawns recorded in Esquimalt Harbour and Esquimalt Lagoon
(last observed in 1993 and 1950, respectively).

\item Spawn recorded at Kiddie Point and Blubber Bay
on Texada Island (first records at these locations).

\item The majority of spawning occurred between March \nth{2} and \nth{7}
after multiple days of high winds and rough seas.

\item The overall length of spawn observed during the flight program
was below average when compared with recent years.

\item Streamkeepers installed spawning panels in False Creek again this year.

\item Overall fishery timing was average this year.

\item The sea lion population in the Strait of Georgia
during the herring season continues to impact the ability
to obtain biological samples in some areas and times.

\end{itemize}