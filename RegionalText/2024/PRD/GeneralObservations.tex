\subsection{Big Bay}

\begin{itemize}

\item First spawn of the season at Tree Bluff on March \nth{15}.
This is the earliest spawn on record.

\item Approximately 15.71 nm of spawn.
Main focus of the spawn concentrated from Big Bay South to Duncan Bay.

\item Average of 3 layers of spawn on substrate.

\end{itemize}

\subsection{Kitkatla}

\begin{itemize}

\item Abundant large herring noticed on the first day of the charter
(March \nth{16}), and the Viking Spirit completed a test set and sample
on the same day with roe yield already at 11.5\%.
Copious fish around Freeman, and spawn first noticed March \nth{19}
(about 3000t) from Cape George to Joachim Rock and Joachim Pt to Goschen Spit.

\item Observed more spawn at Fan Point on March \nth{20},
but none observed in Oval Bay.

\item First flight was on March \nth{21} reported spot spawns at
Kitkatla Creek and Willis Bay and the entire outside coastline had spawn.

\item Freeman spawn still continuing on March \nth{27} (Day 8)
when Viking Spirit departed;
spawn and fish reported in Edye Pass, Welcome Harbour, and Secret Cove.

\item Second flight on Apr \nth{4} reported spot spawns in
Willis Bay, Chief Point, and outside along
Oval Point, William Island, and Henry Island (Welcome Harbour).

\item Overall estimate for the season is between 9 and 10k tonnes
of herring in Area 5.
The Freeman spawn occurred for about 10 days.
Spawn not as strong as in 2023.

\end{itemize}