\subsection{Haida}

In Louscoone Inlet, local herring stocks would normally be found
well up inside the inlet and would often present a very light spawn
near the head of the inlet early in the season (early to mid-March).
This was most likely the case in 2024.
Herring that were sounded and sampled in Louscoone Inlet on March \nth{13}
(estimated $1,400$ tons) may have been part of a migratory stock.
There was a spawn reported on March \nth{22} up toward the head of the inlet
along the west shore which was believed to be very light.

The majority of herring stocks in the Haida Gwaii major stock assessment area
continue to be located around Burnaby Island
(in Skincuttle Inlet and upper Burnaby Strait).
The abundance of stocks and deposition of spawns appear to be
very similar to past years (2022 and 2023).
These stocks appear to be maintaining themselves but
are not really showing any signs of meaningful growth
in spite of no active fisheries for over 20 years.

The Juan Perez Sound inlets normally support spawns
when there are a lot of new “young of the year” in the population.
There were no spawns observed in the upper Juan Perez Sound inlets
during the 2024 season.

In most years Atli Inlet is found to support a small stock of herring.
Such was the case in 2024 with $400$ tons sounded on March \nth{30} and
a light spawn in Beljay Bay in mid-April. 

Herring stocks in Selwyn Inlet continue to show declining trends.
In recent years (2021 and 2023),
the spawns in Selwyn Inlet occurred very late in the season (end of April) and
were very light in intensity.
Herring stocks in Selwyn Inlet remain at a low level of productivity
with no real growth in the population.
There was only a small, light spawn
(about 150m length of trace spawn on eelgrass) observed/reported in early April
in front of Traynor Creek
(adding to concerns that the Selwyn Inlet herring stocks
may be in a serious and irreversible decline).

Herring spawns in Cumshewa Inlet have been infrequent in recent years.
There was a small spawn (about 350m in length) observed/reported at Grey Point,
just outside of Cumshewa Inlet, during the 2024 season.

Approximately $6,500$ lbs of k’aaw was harvested
from the major stock assessment area of East Moresby
(primarily upper Burnaby Strait and Newberry Cove) in early/mid-April.
Product was reported to be moderate to very good in quality.
Most of the traditional k’aaw harvest was conducted by three Haida individuals,
however, much of the product was distributed throughout the
Skidegate and Old Massett communities.
There have been no known harvests of k’aaw from other locations on Haida Gwaii
(i.e., Selwyn Inlet or Skidegate Inlet).
