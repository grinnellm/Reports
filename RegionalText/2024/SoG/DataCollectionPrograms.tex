Biological samples were collected by the seine charter vessel
\boat{Denman Isle} for 27 days from February \nth{20} to March \nth{17}.
Three additional industry test vessels collected biological samples
from March \nth{1} to March \nth{14}.
The primary purpose of the test charter vessel was to
collect biological samples from main aggregations of herring in the SoG,
as identified from soundings within the SAR boundary.

Herring spawn locations were primarily identified with
fixed-wing overflights conducted by DFO Resource Management and Science staff.
Fourteen flights were conducted this season, all in March.

Three dive vessels operated in the SoG:

\begin{itemize}

\item The \boat{Pacific Discovery} surveyed 21 days from March \nth{20} to April \nth{9},

\item The \boat{Ocean Cloud} surveyed 12 days from March \nth{20} to March \nth{31}, and

\item Divers with A-Tlegay Fisheries Society surveyed 1 day on March \nth{13}.

\end{itemize}

The two dive vessels and the seine charter vessel \boat{Denman Isle}
were funded by DFO,
through a contract to the Herring Conservation Research Society.
Additional sampling and sounding efforts conducted through the
Industry Test Program were contributed in-kind by the Herring Industry.
