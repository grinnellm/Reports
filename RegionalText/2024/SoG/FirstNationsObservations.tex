\subsection{Homalko}

Weather delayed the trip to Bute Inlet.
Cruised the East shoreline from Alpha Bluff to Bear River
on Tuesday March \nth{12}th but no definitively white milt in water and
no herring predators observed.
Found the small spawn (about 170m) reported by Bute Inlet Resort
on March \nth{8} on either side of their dock, but
the spawn had stopped some time before we arrived on March \nth{12}.
No milt and no herring observed, just a very few birds and one otter.
The spawn was spotty and about 2-3m wide in rock weed, as well as
on a small patch of grass and on rocks.
No spawning activity observed on the western shore
from Purcell Point back to Orford Bay.


\subsection{Qualicum}

Spawns observed this year were longer and narrow and for the most part
for shorter periods of time.
Early and late spawns may be insurance against environmental disasters.
These long skinny spawns seemed like fish
were looking for optimum places to spawn.
Could also be lack of substrate (e.g., seaweed) to spawn on.
Fish started and continued to spawn in the shallows on ebb tides.
% Would be interesting to compare this years samples with 
% previous year to compare male to female ratio
% and spawn out compared to unspawned fish.
There appeared to be deep water spawning
with milt floating to surface along the edge of the spawn.
Observed spawn was about half of what was observed last year,
taking into consideration the narrow spawns and short duration spawns.
% It would be interesting to compare this with dive survey results.

\begin{itemize}

\item March \nth{6} active spawn off southern Cortez Island (100m by 15m);
several schools in 20 to 40 fathoms (50 to 100 tons);
washed out spawn along shallows on Southeast shore of Cortez Island (1.5 miles);
collected a genetics sample in three fathom;
sounded 1,000 to 1,500 tons in deep water on bottom;
sounded 200 to 300 tons on the bottom in the horseshoe off Cape Lazo; and
sounded 500 tons while crossing the bar heading to Henry Bay.

\item March \nth{7}: no fish in Baynes Sound; and
sounded 300 to 400 tons in upper middle Lambert Channel.

\item March \nth{8}: sounded a few fish in deep hole at top end of Baynes Sound.

\item March \nth{10}: a few fish on sounder in lower Baynes Sound.

\item March \nth{11}: a few fish on sounder off Bowser Shore and Nile Creek; and
large school of herring one mile North of French Creek in 20 to 40 fathoms
(estimate $20,000$ to $25,000$ tons).

\item March \nth{13}: a few schools in fathoms south French Creek Harbour;
light spawn starting one mile north Little Qualicum River and 
more fish on the bottom in 20 fathoms; 
collected a genetics samplein two fathoms;
spawn continuing and growing although the tide was ebbing; and
more fish off Nile Creek and Bowser shore.

\item March \nth{14}: cloud of milt one mile South
of the horseshoe in 20 fathoms;
sounded 10 to 15 tons in three small schools;
collected genetics sample South of Denman Island ferry in two fathoms;
sounded 500 tons in small schools in the middle of Lambert Channel;
spawn starting at Hornby Island from the ferry landing to Ford's Cove;
small spot spawns on Bowser shore and 2.5 miles North of Little Qualicum River;
spawn out to six fathoms with lots of backing fish to 20 fathoms; and
collected genetics sample in in three fathoms.

\item March \nth{15}: spawn from French Creek to mouth of Englishman River,
thickest between harbour and north side Parksville Bay;
collected genetics sample in 3.5 fathoms; and
collected a live sample and delivered it to French Creek.

\item March \nth{16}: sounded three schools of fish
outside of Mistaken Island (250 tons);
a few fish sounded in Nuttle Bay and off Icarus Point in 20 fathoms;
narrow strip of spawn on beach outside of Keel Cove; and
collected genetics sample; in 2.5 fathoms.

\item March \nth{17}: fish on bottom in 5 to 20 fathoms off Icarus Point and
washed out spawn all along beach;
a few fish and washed out spawn in Nuttle  Bay;
a few fish at Blunden Point;
spawn on beach off Lantzville and South of Icarus Point;
collected a genetics sample;

\item March \nth{18}: spots of washed out spawn off Lantzville;
a few fish sounded on the bottom in Pylades Channel and outside Silva Bay;

\item March \nth{19}: a few fish moving around Outside islands of Silva Bay;
spawn starting eastern side of Tugboat Island and Sears Island;
collected genetics sample; and
narrow spawn in Logan Bay.

\end{itemize}

\subsection{Tla'amin}

The Nation's fishing crew for herring encountered a few light spawn sites on the
southwest side of Hernando Island and the northeast side of Cortis Island.
There was lots of milky water but no evidence of any eggs.
A few crew members got off the boat to check the beach for eggs
but none were observed. 
Seems that there was a higher ratio of males in the mix.
The Nation was unable to get any herring or roe
from Area 15 this year despite multiple efforts
