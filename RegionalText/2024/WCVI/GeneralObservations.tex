\begin{itemize}

\item Relatively high numbers of cast-net samples (23)
collected by First Nation charters in Areas 23, 24, and 25 (Esperanza).

\item Difficulty obtaining biological samples in Area 25 (Nootka)
was due to poor weather, erratic fish movement, and
fish being either off-shore, later, or too close to the beaches.

\item Early spawns occurring before March \nth{1} were not dive surveyed
due to dive resource limitations.
These include Kraan Head and Hesquiat Harbour in Area 24,
and Nuchatlitz in Area 25.

\item Spawns occurring on South Vargas Island were not surveyed
due their exposed nature and inclement weather.

\item In-season communication between all parties
contributed to a successful spawn survey.

\item Anecdotal reports suggest increased success in FSC harvest.

\item Spawn distribution was widespread and
included new locations when compared to previous years:
North of Big Beach and Wya Point in Area 23,
northwest and southwest shores of Vargas Island in Area 24,
Zuciarte Channel in Area 25 (Nootka),
Maquinna Point in Area 25 (Nootka),
Resolution Cove and Pantoja Islands in Area 25 (Nootka),
mouth of Port Eliza Inlet in Area 25 (Esperanza), and
Mary Basin in Area 25 (Esperanza).

\item Spawn was reported inside of net pens at an
aquaculture site in Fortune Channel in Area 24

\end{itemize}