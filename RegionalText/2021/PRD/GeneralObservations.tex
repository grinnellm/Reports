\begin{itemize}
\item Continued COVID-19 precautions meant PRD Resource Management staff were
not able to be on-grounds during the assessment activities.
\item High winds and inclement weather throughout the test fishery were a challenge for charter operators.
\item Large bodies of fish were observed in Kitkatla Inlet, but
moved out and no spawn was observed in the typical locations at the top of the inlet.
\item A significant spawn occurred on the outside of Kitkatla Inlet
towards Joachim point and in front of Kitkatla Village.
\item Water sea surface temperatures were similar to last year (5 to 6 degrees) and colder than recent years.
\item Initial spawn occurred later than the last few years possibly due to the colder water temperatures.
\item Egg deposition was better than average throughout both assessment areas.
\item Collaboration on spawn surveys with local First Nations fishery programs
greatly increased the capacity to assess the areas.

\item Spawning was later due poor weather conditions and colder water temperatures.
\item Spawn occurred simultaneously throughout the assessment area.
\item Overall length of spawn was below average.
\item Good layers (6 to 8) observed from Big Bay south to Ryan Pt.
\item No spawning observed at the north end of Porcher or Tugwell Island.
\item Due to the late spawn,
Lax Kw'alaams and Metlakatla were key in mapping and recording spawn for dives.
\end{itemize}