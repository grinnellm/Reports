Biological samples were collected by the seine charter vessel "Proud Canadian"
for 21 days from February \nth{20} to March \nth{12}.
The primary purpose of the test charter vessel was to collect biological samples from
main bodies of herring from Statistical Areas 23, 24, and 25.
Nearshore herring samples were collected by the Nuu-chah-nulth staff
as part of a sampling program and on-going collaboration between WCVI First Nations and DFO.
These nearshore biological samples were collected from spawning aggregations using cast nets.
WCVI First Nations spawn reconnaissance charters reported spawn activity in all three areas.
Herring spawn locations were primarily identified with fixed-wing overflights conducted by
DFO Resource Management Area staff.
Thirteen flights were conducted this season in February and March.

Two dive charter vessels operated in WCVI:
\begin{itemize}
\item The "Pachena No.1" surveyed 15 days from March \nth{18} to April \nth{1}, and
\item The "Seaveyor" surveyed 10 days from March \nth{20} to April \nth{3}.
\end{itemize}
The seine test charter vessel and the dive survey vessels were funded by DFO,
through a contract to the Herring Conservation Research Society.
